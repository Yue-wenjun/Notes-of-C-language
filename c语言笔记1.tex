\documentclass{book}
\usepackage{newtxtext}
\usepackage[marginparwidth={4cm},lmargin={1cm}, rmargin={5cm}]{geometry}
\usepackage[dvipsnames,svgnames,x11names]{xcolor}
\usepackage[strict]{changepage} % 提供一个 adjustwidth 环境
\usepackage{framed} % 框包
\usepackage{xeCJK}%中文包
\usepackage{amsmath}%写连等式对齐包
\usepackage{graphicx}%插入图片
\usepackage{wallpaper}%插入背景图
\usepackage{indentfirst}%latex第一段首行不缩进,看着不爽,于是用包使每段都不缩进
%\usepackage{titlesec}如有需要设置各级标题
%\usepackage{ctex}如有需要设置中文字体


\geometry{a4paper,centering,scale=0.8}
\definecolor{formalshade}{rgb}{0.95,0.95,1} % 文本框颜色
% ------------------******-------------------
% 注意行末需要把空格注释掉,不然画出来的方框会有空白竖线
\definecolor{greenshade}{rgb}{0.92,1,0.92}
\definecolor{grayshade}{rgb}{0.90,0.90,0.90}
%蓝紫框--------------------------------------------
\newenvironment{formal1}{%
\def\FrameCommand{%
\hspace{1pt}%
{\color{DarkBlue}\vrule width 2pt}%
{\color{SteelBlue}\vrule width 4pt}%
\colorbox{formalshade}%
}%
\MakeFramed{\advance\hsize-\width\FrameRestore}%
\noindent\hspace{-4.55pt}% disable indenting first paragraph
\begin{adjustwidth}{}{7pt}%
\vspace{2pt}\vspace{2pt}%
}
{%
\vspace{2pt}\end{adjustwidth}\endMakeFramed%
}
%绿框---------------------------------------
\newenvironment{formal2}{%
\def\FrameCommand{%
\hspace{1pt}%
{\color{Green}\vrule width 2pt}%
{\color{DarkSeaGreen1}\vrule width 4pt}%
\colorbox{greenshade}%
}%
\MakeFramed{\advance\hsize-\width\FrameRestore}%
\noindent\hspace{-4.55pt}% disable indenting first paragraph
\begin{adjustwidth}{}{7pt}%
\vspace{2pt}\vspace{2pt}%
}
{%
\vspace{2pt}\end{adjustwidth}\endMakeFramed%
}
%灰框---------------------------------------
\newenvironment{formal3}{%
\def\FrameCommand{%
\hspace{1pt}%
{\color{darkgray}\vrule width 2pt}%
{\color{Ivory4}\vrule width 4pt}%
\colorbox{grayshade}%
}%
\MakeFramed{\advance\hsize-\width\FrameRestore}%
\noindent\hspace{-4.55pt}% disable indenting first paragraph
\begin{adjustwidth}{}{7pt}%
\vspace{2pt}\vspace{2pt}%
}
{%
\vspace{2pt}\end{adjustwidth}\endMakeFramed%
}
%天蓝框---------------------------------------
\newenvironment{formal4}{%
\def\FrameCommand{%
\hspace{1pt}%
{\color{SkyBlue}\vrule width 2pt}%
{\color{Cyan1}\vrule width 4pt}%
\colorbox{LightCyan1}%
}%
\MakeFramed{\advance\hsize-\width\FrameRestore}%
\noindent\hspace{-4.55pt}% disable indenting first paragraph
\begin{adjustwidth}{}{7pt}%
\vspace{2pt}\vspace{2pt}%
}
{%
\vspace{2pt}\end{adjustwidth}\endMakeFramed%
}
%粉框---------------------------------------
\newenvironment{formal5}{%
\def\FrameCommand{%
\hspace{1pt}%
{\color{HotPink}\vrule width 2pt}%
{\color{LightPink}\vrule width 4pt}%
\colorbox{MistyRose}%
}%
\MakeFramed{\advance\hsize-\width\FrameRestore}%
\noindent\hspace{-4.55pt}% disable indenting first paragraph
\begin{adjustwidth}{}{7pt}%
\vspace{2pt}\vspace{2pt}%
}
{%
\vspace{2pt}\end{adjustwidth}\endMakeFramed%
}
%首段首行缩进--------------------------------------
\setlength{\parindent}{0em}

% ------------------******-------------------
% ------------------******-------------------
% ------------------******-------------------
% ------------------******------------------
\begin{document}
 \begin{titlepage}
 \newgeometry{left=3cm,right=3cm}
 \ThisCenterWallPaper{1.255}{8.jpg}
 \color{darkgray} 
 \begingroup
 \thispagestyle{empty}
 \centering
 \vspace*{5cm}
 \par\normalfont\fontsize{35}{35}\sffamily\selectfont
 \textbf{C/C++}\\ 
 {\LARGE 001 environment settings in vscode}\par % Book title
 \vspace*{1cm}
 {\Huge Notes of Practice}\par % Author name
 \endgroup
 \end{titlepage}



 \section{操作步骤}
\marginpar{下载Mingw-w64要挂梯子\\
没有梯子就在网上找网盘\\
\\
将文件下载到没有中文和空格的路径(重点)\\
\\
电脑path选账户环境变量,防止系统崩掉\\
\\
vscode配置的有点多,下方安利一个程序\\
}


\par \begin{enumerate}
 \item 下载Mingw-w64(就是它的版本问题,不要点大download,即10.0.0版,在file里找一个8.10.0版本即可)
 \item 下载7z(官网上下载的需要解压)
 \item 添加环境变量path
 \item 检验配置
 \item 下载vscode插件
 \item 配置vscode
\end{enumerate}

\color{darkgray}
 \definecolor{shadecolor}{rgb}{0.90,0.9,0.90}
 \begin{shaded}
 {\subsection[short]{下载内容}}
\end{shaded}
\color{black}

\begin{formal4}
以下分别是mingw-w64和7z的官网\\    
\begin{itemize}
\item https://sourceforge.net/projects/mingw-w64/files/
\item https://www.7-zip.org/download.html\\
\end{itemize}
再给大家安利一个程序,可以帮你配置好vscode,一次没配好还可以删了重配\\
一开始是自己慢慢敲的,结果要反复调,用了这个就很省事。\\
VS Code Config Helper,链接https://v4.vscch.tk/,免费真香。
\end{formal4}

\color{darkgray}
 \definecolor{shadecolor}{rgb}{0.90,0.9,0.90}
 \begin{shaded}
 {\subsection[short]{电脑配置内容}}
\end{shaded}
\color{black}

\begin{formal2}
\begin{itemize}
\item 配置
\end{itemize}
在mingw64里找到文件夹“bin”,复制路径\\
在windows里搜索“账户环境变量”点进去,找到Path,打开在下面加一行,把复制的路径粘贴进去\\
如果系统不一样,是写一排的,就打英文分号补后面\\
\begin{itemize}
\item 检验
\end{itemize}
打开cmd,输入“where gcc”后弹出路径\\
输入“gcc --version”后弹出gcc版本信息
\end{formal2}

\color{darkgray}
 \definecolor{shadecolor}{rgb}{0.90,0.9,0.90}
 \begin{shaded}
 {\subsection[short]{vscode配置内容}}
\end{shaded}
\color{black}

\marginpar{代码篇幅太长,写不下\\
}

\begin{formal1}
\begin{itemize}
\item 废话
\end{itemize}
把vscode和C/C++插件下好,插件会帮你建“.vscode”文件夹,里面有三个把它移到你想要的文件夹里,然后修改里面的文件。
有个版本甚至没有帮你建其中一个文件,那就自己建一个.\\
\begin{itemize}
\item 代码
\end{itemize}
https://zhuanlan.zhihu.com/p/363798930,这篇文章拉到最下面复制粘贴,把launch.json的路径改成自己的。
\end{formal1}

\section{网络教程}
\marginpar{多看看官网总是好的\\
}
\begin{itemize}
    \item https://zhuanlan.zhihu.com/p/77074009(小白教程)
    \item https://code.visualstudio.com/docs/cpp/config-mingw(官方教程很详细,但是实操起来缺少点细节)
    \item https://code.visualstudio.com/docs/cpp/cpp-debug(官方debug教程)
\end{itemize}

\section{关于Dev-C++}
为什么我在有这么方便统一的软件情况下还是选择配置vscode。
\begin{itemize}
    \item 首先,我是颜狗lol
    \item 不止会用C,我还要用其他语言,把各种语言放在一起写很方便
    \item 学习新技能
    \item vscode yyds
\end{itemize}
但还是要学习dev的,毕竟不是每个环境、每个考场都有vs
\begin{formal2}
最后,既然开了个头,那就把我以后的C语言笔记都放在里面吧
\end{formal2}

\end{document}